\documentclass[]{article}
\usepackage{lmodern}
\usepackage{amssymb,amsmath}
\usepackage{ifxetex,ifluatex}
\usepackage{fixltx2e} % provides \textsubscript
\ifnum 0\ifxetex 1\fi\ifluatex 1\fi=0 % if pdftex
  \usepackage[T1]{fontenc}
  \usepackage[utf8]{inputenc}
\else % if luatex or xelatex
  \ifxetex
    \usepackage{mathspec}
  \else
    \usepackage{fontspec}
  \fi
  \defaultfontfeatures{Ligatures=TeX,Scale=MatchLowercase}
\fi
% use upquote if available, for straight quotes in verbatim environments
\IfFileExists{upquote.sty}{\usepackage{upquote}}{}
% use microtype if available
\IfFileExists{microtype.sty}{%
\usepackage{microtype}
\UseMicrotypeSet[protrusion]{basicmath} % disable protrusion for tt fonts
}{}
\usepackage[margin=1in]{geometry}
\usepackage{hyperref}
\hypersetup{unicode=true,
            pdftitle={Reinforcement Learning Basics Part 1 -},
            pdfauthor={miero},
            pdfborder={0 0 0},
            breaklinks=true}
\urlstyle{same}  % don't use monospace font for urls
\usepackage{graphicx,grffile}
\makeatletter
\def\maxwidth{\ifdim\Gin@nat@width>\linewidth\linewidth\else\Gin@nat@width\fi}
\def\maxheight{\ifdim\Gin@nat@height>\textheight\textheight\else\Gin@nat@height\fi}
\makeatother
% Scale images if necessary, so that they will not overflow the page
% margins by default, and it is still possible to overwrite the defaults
% using explicit options in \includegraphics[width, height, ...]{}
\setkeys{Gin}{width=\maxwidth,height=\maxheight,keepaspectratio}
\IfFileExists{parskip.sty}{%
\usepackage{parskip}
}{% else
\setlength{\parindent}{0pt}
\setlength{\parskip}{6pt plus 2pt minus 1pt}
}
\setlength{\emergencystretch}{3em}  % prevent overfull lines
\providecommand{\tightlist}{%
  \setlength{\itemsep}{0pt}\setlength{\parskip}{0pt}}
\setcounter{secnumdepth}{0}
% Redefines (sub)paragraphs to behave more like sections
\ifx\paragraph\undefined\else
\let\oldparagraph\paragraph
\renewcommand{\paragraph}[1]{\oldparagraph{#1}\mbox{}}
\fi
\ifx\subparagraph\undefined\else
\let\oldsubparagraph\subparagraph
\renewcommand{\subparagraph}[1]{\oldsubparagraph{#1}\mbox{}}
\fi

%%% Use protect on footnotes to avoid problems with footnotes in titles
\let\rmarkdownfootnote\footnote%
\def\footnote{\protect\rmarkdownfootnote}

%%% Change title format to be more compact
\usepackage{titling}

% Create subtitle command for use in maketitle
\providecommand{\subtitle}[1]{
  \posttitle{
    \begin{center}\large#1\end{center}
    }
}

\setlength{\droptitle}{-2em}

  \title{Reinforcement Learning Basics Part 1 -}
    \pretitle{\vspace{\droptitle}\centering\huge}
  \posttitle{\par}
    \author{miero}
    \preauthor{\centering\large\emph}
  \postauthor{\par}
      \predate{\centering\large\emph}
  \postdate{\par}
    \date{2020-01-30}

\usepackage{notation}
\usepackage[latin1]{inputenc}
\usepackage{tikz}
\usetikzlibrary{shapes,arrows}

\begin{document}
\maketitle

\hypertarget{introduction}{%
\paragraph{Introduction}\label{introduction}}

I will first give an overview of what Reinforcement Learning (RL) is and
then introduce the subject by providing analogies to the concept of RL.
Then provide definitions, explain some of the basic Mathematics, and
finally some python code to go along with it.

\hypertarget{overview}{%
\paragraph{Overview}\label{overview}}

\tikzstyle{decision} = {[}diamond, draw, fill=blue!20, text width=4.5em,
text badly centered, node distance=3cm, inner sep=0pt{]}
\tikzstyle{block} = {[}rectangle, draw, fill=blue!20, text width=5em,
text centered, rounded corners, minimum height=4em{]} \tikzstyle{line} =
{[}draw, -latex'{]} \tikzstyle{cloud} = {[}draw, ellipse,fill=red!20,
node distance=3cm, minimum height=2em{]}

\begin{tikzpicture}[node distance = 2cm, auto]

    % Place nodes
    \node [block] (env) {World to Interact With};
    \node [cloud, below of=env] (agent) {Agent/Gamer};
    
    \draw[->] (env.east) -- ++(0.65,0) -- ++(0,0) -- ++(0,0) --                
     node[xshift=0.1cm,yshift=0cm, text width=2.5cm]
     {$s_{t}, r_{t}$ }(agent.east);
    
     \draw[->] (agent.west) -- ++(0,3) -- ++(0,0) -- ++(0,0) --                
     node[xshift=0.35cm,yshift=-1.75cm, text width=2.5cm]
     {$a_{t}$}(env.west);
    
\end{tikzpicture}

\hypertarget{analogies}{%
\paragraph{Analogies}\label{analogies}}


\end{document}
