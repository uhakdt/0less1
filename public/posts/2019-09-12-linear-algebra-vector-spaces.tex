\documentclass[]{article}
\usepackage{lmodern}
\usepackage{amssymb,amsmath}
\usepackage{ifxetex,ifluatex}
\usepackage{fixltx2e} % provides \textsubscript
\ifnum 0\ifxetex 1\fi\ifluatex 1\fi=0 % if pdftex
  \usepackage[T1]{fontenc}
  \usepackage[utf8]{inputenc}
\else % if luatex or xelatex
  \ifxetex
    \usepackage{mathspec}
  \else
    \usepackage{fontspec}
  \fi
  \defaultfontfeatures{Ligatures=TeX,Scale=MatchLowercase}
\fi
% use upquote if available, for straight quotes in verbatim environments
\IfFileExists{upquote.sty}{\usepackage{upquote}}{}
% use microtype if available
\IfFileExists{microtype.sty}{%
\usepackage{microtype}
\UseMicrotypeSet[protrusion]{basicmath} % disable protrusion for tt fonts
}{}
\usepackage[margin=1in]{geometry}
\usepackage{hyperref}
\hypersetup{unicode=true,
            pdftitle={Linear Algebra: Vector Spaces},
            pdfauthor={miero},
            pdfborder={0 0 0},
            breaklinks=true}
\urlstyle{same}  % don't use monospace font for urls
\usepackage{graphicx,grffile}
\makeatletter
\def\maxwidth{\ifdim\Gin@nat@width>\linewidth\linewidth\else\Gin@nat@width\fi}
\def\maxheight{\ifdim\Gin@nat@height>\textheight\textheight\else\Gin@nat@height\fi}
\makeatother
% Scale images if necessary, so that they will not overflow the page
% margins by default, and it is still possible to overwrite the defaults
% using explicit options in \includegraphics[width, height, ...]{}
\setkeys{Gin}{width=\maxwidth,height=\maxheight,keepaspectratio}
\IfFileExists{parskip.sty}{%
\usepackage{parskip}
}{% else
\setlength{\parindent}{0pt}
\setlength{\parskip}{6pt plus 2pt minus 1pt}
}
\setlength{\emergencystretch}{3em}  % prevent overfull lines
\providecommand{\tightlist}{%
  \setlength{\itemsep}{0pt}\setlength{\parskip}{0pt}}
\setcounter{secnumdepth}{0}
% Redefines (sub)paragraphs to behave more like sections
\ifx\paragraph\undefined\else
\let\oldparagraph\paragraph
\renewcommand{\paragraph}[1]{\oldparagraph{#1}\mbox{}}
\fi
\ifx\subparagraph\undefined\else
\let\oldsubparagraph\subparagraph
\renewcommand{\subparagraph}[1]{\oldsubparagraph{#1}\mbox{}}
\fi

%%% Use protect on footnotes to avoid problems with footnotes in titles
\let\rmarkdownfootnote\footnote%
\def\footnote{\protect\rmarkdownfootnote}

%%% Change title format to be more compact
\usepackage{titling}

% Create subtitle command for use in maketitle
\providecommand{\subtitle}[1]{
  \posttitle{
    \begin{center}\large#1\end{center}
    }
}

\setlength{\droptitle}{-2em}

  \title{Linear Algebra: Vector Spaces}
    \pretitle{\vspace{\droptitle}\centering\huge}
  \posttitle{\par}
    \author{miero}
    \preauthor{\centering\large\emph}
  \postauthor{\par}
      \predate{\centering\large\emph}
  \postdate{\par}
    \date{2019-09-12}


\begin{document}
\maketitle

We will discuss the vector space initially as an abstract notion before
diving into examples and then some further explanations.

\(Let\ V\ be\ a\ non-empty\ set.\ Let\ \boldsymbol{a, b,c }\ \in\ V\ and\ \lambda\ , \mu\ \in\ \mathbb{R}.\)
We will denote \(\oplus\) as `sum' and \$\circ\$ as `scalar
multiplication'. V now satisfies the following axioms:

\begin{align}
 &(i)\ \ \ \ a\oplus b \in V \ \                                                               &[the\ sum\ of\ \boldsymbol{a}\ and\ \boldsymbol{b}\ is\ in\ V] \\\\
 &(ii)\ \ \ \ (a\oplus b)\oplus c= a\oplus (b\oplus c)\                                        &[the\ \boldsymbol{associativity\ law}\ holds\ for\ \ \oplus] \\\\
 &(iii)\ \ \ \ \exists\ e\ \in\ V\ \mid\ a\oplus e = e\oplus a = a\                            &[Existence\ of\ the\ \boldsymbol{identity\ element}] \\\\
 &(iv)\ \ \ \ \forall\ a\ \in\ V\ there\ exists\ b\ \in\ V\ \mid\ a\oplus b = b\oplus a = e\   &[Existence\ of\ the\ \boldsymbol{right\ \&\ left\ Inverse}]\\\\
 &(v)\ \ \ \ a\oplus b = b\oplus a\                                                            &[the\ \boldsymbol{commutativity\ law}\ holds\ for\ \ \oplus]\\\\
 &(vi)\ \ \ \ \lambda \circ a\ \in\ V\                                                         &[scalar\ multiples\ \in\ V]\\\\
 &(vii)\ \ \ \ \lambda \circ (a\oplus b)=\lambda \circ a\oplus \lambda \circ b                 &[1^{st}\ \boldsymbol{distributivity\ law}\ holds]\\\\
 &(viii)\ \ \ \ \lambda + \mu \circ a = \lambda \circ a\oplus \mu \circ a                      &[2^{nd}\ \boldsymbol{distributivity\ law}\ holds]\\\\
 &(ix)\ \ \ \ \lambda \circ (\mu \circ a) = (\lambda\mu)\circ a                                &[the\ \boldsymbol{associativity\ law}\ holds\ for\ \ \circ]\\\\
 &(x)\ \ \ \ 1 \circ a = a                                                                     &[Existence\ of\ the\ \boldsymbol{identity\ element}\ for\ \ \circ]\\\\
\end{align}

These 10 axioms look very similar to axioms that would hold for Groups
and Fields, but we will discuss these in later posts. The following
examples all hold for all of the axioms above: Example 1:

References:


\end{document}
